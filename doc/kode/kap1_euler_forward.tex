%%%%%%%%%%%%%%%%%%%%%%%%%%%%%%%%%%%%%%%%%%%%%%%%%%%%%%%%%%%%%%%
% KAPPITTEL 1: KODE EULER FORWARD
%%%%%%%%%%%%%%%%%%%%%%%%%%%%%%%%%%%%%%%%%%%%%%%%%%%%%%%%%%%%%%%

\documentclass[../main.tex]{subfiles}

%%%%%%%%%%%%%%%%%%%%%%%%%%%%%%%%%%%%%%%%%%%%%%%%%%%%%%%%%%%%%%%
% Start av dokumentet
%%%%%%%%%%%%%%%%%%%%%%%%%%%%%%%%%%%%%%%%%%%%%%%%%%%%%%%%%%%%%%%

\begin{document}

\begin{minipage}[htb]{\textwidth}

    \begin{lstlisting}[%
    style=Matlab-editor,
    basicstyle=\footnotesize,
    label={lst:kap1_euler},
    caption={Generell implementering av Euler's forovermetode i \textsc{MATLAB}.}
    ]
for k = 1 : n
    if k == 1
        y(k) = 0; % Initialverdi
    else
        y(k) = y(k-1) + delta_x * f(k-1); % Rekursiv integrering
    end
end
\end{lstlisting}

\end{minipage}

\end{document}