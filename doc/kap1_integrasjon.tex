%%%%%%%%%%%%%%%%%%%%%%%%%%%%%%%%%%%%%%%%%%%%%%%%%%%%%%%%%%%%%%%
% KAPPITTEL 1: NUMERISK INTEGRASJON
%%%%%%%%%%%%%%%%%%%%%%%%%%%%%%%%%%%%%%%%%%%%%%%%%%%%%%%%%%%%%%%

\documentclass[main.tex]{subfiles}

%%%%%%%%%%%%%%%%%%%%%%%%%%%%%%%%%%%%%%%%%%%%%%%%%%%%%%%%%%%%%%%
% Start av dokumentet
%%%%%%%%%%%%%%%%%%%%%%%%%%%%%%%%%%%%%%%%%%%%%%%%%%%%%%%%%%%%%%%

\begin{document}

\chapter{Numerisk integrasjon (utført av hele gruppen) }\label{kap:integrasjon}

\section{Problemstilling}
Gruppen ønsker å vise at måledata fra \textsc{EV3}-roboten kan numerisk integreres ved bruk av \textsc{MATLAB}.

\section{Numerisk integrasjon}
Numerisk integrasjon er en metode for å estimere et bestemt integral. Metoden bygger på at integranden $f(x)$ er en serie med diskrete verdier. Ved å kalkulere arealet under $f(x)$ for hvert intervall i serien og summere disse er det mulig å estimere det bestemte integralet \parencite[s. 297–301]{AdaEss2017}. Et estimat for arealet under integranden $f(x)$ kan derfor skrives som en sum over alle arealene på intervallene i serien

\begin{equation}\label{eq:kap1_sum_area}
    \hat{A} =\sum_{k=1}^{n-1} \, f(x_k) \cdot \Delta x_k
\end{equation}

hvor $\Delta x_k$ er lengden på intervallet mellom to målinger og $x_k$ er verdiene til integranden ved de respektive intervallet. Figur~\ref{fig:kap1_partitioning} viser ligning \eqref{eq:kap1_sum_area} plottet mot funksjonen $f$.

\subfile{./figurer/kap1_partitioning.tex}

For å kunne benytte ligning \eqref{eq:kap1_sum_area} i \textsc{MATLAB} må den skrives på rekursiv form \parencite[s.~53–58]{Dre2023Simulink}. Vi har valgt å benytte Euler's forovermetode som har form
\begin{equation}\label{eq:kap1_area_rec}
    y_k = y_{k-1} + \Delta{x_k} \cdot f(x_{k-1}) \quad \forall k=2,..., n
\end{equation}
hvor $k$ er indeksen til gjeldende måling, $y_k$ er arealet under itegranden ved gjeldende måling, $y_{k-1}$ er arealet under integranden ved forrige måling, $\Delta{x_k}$ er tidsskrittet ved gjeldende måling og $f(x_{k-1})$ er verdien til integranden ved forrige måling.

Grunnen til at vi har valgt å benytte Euler's forovermetode at metoden gir et estimat på arealet under integranden på gjeldende intervall gitt verdien av arealet og integranden ved forrige intervall. Dette er ønskelig utifra at vi ønsker å kalkulere gjeldende areal hurtig, og at vi kun må lagre ett areal for å kunne estimere arealet på neste intervall.

En generell implementering av den rekursive ligningen \eqref{eq:kap1_area_rec} i \textsc{MATLAB} kan sees i kode~\ref{lst:kap1_euler}. Legg merke til at en initialverdi for $y(1)$ må settes på linje~3 for å kunne estimere arealet på det første intervallet.

\subfile{./kode/kap1_euler_forward.tex}

\section{Implentering av numerisk integrasjon i \textsc{MATLAB}}

For å verifisere at Euler's forovermetode kan brukes til å integrere måledata i \textsc{MATLAB} har vi utført to forsøk. Begge forsøkene tar utgangspunkt i å simulere en beholder med væske som fylles og tappes. En konseptuell skisse av oppsettet for forsøket kan sees i figur~\ref{fig:kap1_skisse_beholder}.

\subfile{./figurer/kap1_skisse_beholder.tex}

Raten $u(k)$ det fylles og tappes væske i beholderen styres av måledata fra lyssensoren til \textsc{EV3}-roboten. Volumet av væske i beholderen $y(k)$ kalkulereres ved å numerisk integrere fylle- og tapperaten $u(k)$ i \textsc{MATLAB}. Dette gjøres ved å bruke uttrykket i \eqref{eq:kap1_area_rec} og følgende implementering av den generelle koden vist i kode~\ref{lst:kap1_euler}.

\subfile{./kode/kap1_integrasjon_beholder.tex}

For å verifisere at implementeringen vår i \textsc{MATLAB} er riktig har vi sammenlignet resultatene fra numerisk integrasjonen i de to forsøkene med analytiske verdier.

\subsection{Integrasjon av en konstant}\label{sub:kap1_integrasjon_konstant}

Dette forsøket simulerer konstant fylling og tapping av væske i beholderen i figur~\ref{fig:kap1_skisse_beholder}. Fylling og tapping av væske gjennom kranene blir simulert ved å lese av reflektert lys fra en gråskalastripe. Gråskalastripen kan sees i Figur~\ref{fig:kap1_skala}.

\subfile{./figurer/kap1_skala.tex}

Lysmåleren til \textsc{EV3}-roboten blir plasert på gråskalafeltet med metning 50\% i midten av stripen i Figur~\ref{fig:kap1_skala}. Når programmet i \textsc{MATLAB} som simulerer fylling og tapping blir startet blir den første lysmålingen fra lyssensoren gitt verdien $u(0) = \SI{0}{cl/s}$. Ved å flytte lysmåleren til høyre for startfeltet vil strømningsraten bli positiv, og ved å flytte lysmåleren til venstre for startfeltet vil strømningsraten bli negativ.

I figur~\ref{fig:kap1_konstant} er fylle og tapperaten $u(k)$ plottet sammen med volumet $y(k)$ i beholderen.

\subfile{./grafer/kap1_konstant.tex}

For å kontrollere at implementeringen av Euler's fremovermetode er korrekt, kan vi sammenligne resultatene i figur~\ref{fig:kap1_konstant} med et analytisk integral av måledataene. Ved å integrere $u(t)$ fra $t = \SI{1.8}{s}$ til $t = \SI{4.4}{s}$ så skal integralet tilsvare arealet $A1$ under$u(t)$ i tidsintervallet. Arealet $A1$ skal så være lik forandringen i volum $\Delta y$ ved slutten av intervallet i $y(t)$.

\begin{align}
    A1 & = \int_{1.8}^{4.4} \Delta u \, dt \nonumber         \\
       & = \Delta u_1 \cdot t \, \Big|_{1.8}^{4.4} \nonumber \\
       & = \Delta u_1 \cdot (4.4 - 1.8) \nonumber            \\
       & = \SI{52}{cl} \label{eq:kap1_konstant_integral}
\end{align}

Ved å sammeligne verdiene av integralet i \eqref{eq:kap1_konstant_integral} med forandringen i volum $\Delta y$ i figur~\ref{fig:kap1_konstant} kan vi se at resultatene fra numerisk integrasjon er korrekte.

\subsection{Integrasjon av et sinussignal}

Dette forsøket simulerer også fylling og tapping av væske i beholderen skissert i figur~\ref{fig:kap1_skisse_beholder}. Til forskjell fra forsøket beskrevet i seksjon~\ref{sub:kap1_integrasjon_konstant} simuleres nå fylling og tapping som et sinussignal.

Et sinus-signal blir generert ved å bevege lysmåleren til \textsc{EV3}-roboten i en sirkel på et A4-gråskalaark med en jevn gradient fra hvitt til svart. Hvordan lysmåleren beveges på gråskalaarket kan sees i figur~\ref{fig:kap1_skala}.

\subfile{./figurer/kap1_ark.tex}

To målinger ble gjennomført der lysmåleren ble beveget i en sirkel på gråskalaarket med forskjellig hastighet. I figur~\ref{fig:kap1_sinus1} er fylle- og tapperaten $u_1(t)$ plottet sammen med volumet $y_1(t)$ i beholderen for den første hastigheten. Figur~\ref{fig:kap1_sinus2} viser fylle- og tapperaten $u_2(t)$ plottet sammen med volumet $y_2(t)$ i beholderen for den andre hastigheten.

\subfile{./grafer/kap1_sinus1.tex}

\subfile{./grafer/kap1_sinus2.tex}

For å kontrollere implementeringen av Euler's fremovermetode korrekt, kan vi sammenligne måledataene med et analytisk integral av et sinussignal.

Integralet av et sinussignal kan skrives som
\begin{align}
    I & = \int_{0}^{t} a_u \cdot \sin{(\omega \cdot t)} \, dt \nonumber \\
      & = -\frac{a}{\omega} \cdot \cos{(\omega \cdot t)}
\end{align}

hvor vinkelfrekvensen $\omega$ regnes ut fra perioden $T_p$ lest av fra figur~\ref{fig:kap1_sinus1} og \ref{fig:kap1_sinus2}
\begin{equation}
    \omega = \frac{2\pi}{T_p}
\end{equation}

som betyr at amplituden $a_y$ til volumet $y(t)$ i beholderen kan regnes ut fra
\begin{equation}
    a_y = \frac{a_u}{\omega}
\end{equation}

Tabell~\ref{tab:kap1_sinus} viser disse utregningene for lav og høy hastighet på sirkelbevegelsen.

\subfile{./tabeller/kap1_sinus.tex}

Ved a sammenligne verdiene for $a_y$ i tabellen med verdiene lest av fra figur~\ref{fig:kap1_sinus1} og \ref{fig:kap1_sinus2} kan vi konkludere med at implementeringen av den numeriske integrasjonen er korrekt.

\subsection{Integrasjon som funksjon}

\end{document}
