%%%%%%%%%%%%%%%%%%%%%%%%%%%%%%%%%%%%%%%%%%%%%%%%%%%%%%%%%%%%%%%
% KAPPITTEL 1: NUMERISK INTEGRASJON
%%%%%%%%%%%%%%%%%%%%%%%%%%%%%%%%%%%%%%%%%%%%%%%%%%%%%%%%%%%%%%%

\documentclass[main.tex]{subfiles}

%%%%%%%%%%%%%%%%%%%%%%%%%%%%%%%%%%%%%%%%%%%%%%%%%%%%%%%%%%%%%%%
% Start av dokumentet
%%%%%%%%%%%%%%%%%%%%%%%%%%%%%%%%%%%%%%%%%%%%%%%%%%%%%%%%%%%%%%%

\begin{document}

\chapter{Numerisk integrasjon (utført av hele gruppen) }\label{kap:integrasjon}

\section{Problemstilling}
Gruppen ønsker å vise at måledata fra \textsc{EV3}-roboten kan numerisk integreres ved bruk av \textsc{MATLAB}.

\section{Numerisk integrasjon}
Numerisk integrasjon er en metode for å estimere et bestemt integral. Metoden bygger på at integranden $f(x)$ er, eller kan deles opp i, en serie med diskrete intervaller. Arealet under $f(x)$ estimeres som en sum av arealene for hvert intervall i serien \parencite[s. 297–301]{AdaEss2017}.

Utrykket for å estimere arealet under $f(x)$ for en serie diskrete verdier $n$ kan derfor utrykkes som
\begin{equation}\label{eq:kap1_sum_area}
    \hat{A} =\sum_{k=1}^{n-1} \, f(x_k) \cdot \Delta x_k
\end{equation}
hvor $\Delta x_k$ er lengden på intervallet ved element $k$, og $f(x_k)$ er verdien til integranden på intervall $x_k$. Figur~\ref{fig:kap1_partitioning} viser et grafisk eksempel på hvordan ligning \eqref{eq:kap1_sum_area} estimerer arealet under $f(x)$.

\subfile{./figurer/kap1_partitioning.tex}

Etterhvert som lengden på intervallene $\Delta x_k$ blir uendelig små, altså at $\lim_{n \to \infty}$, så vil ligning \eqref{eq:kap1_sum_area} konvergere mot det bestemte integralet for $f(x)$.

For å kunne benytte ligning \eqref{eq:kap1_sum_area} i \textsc{MATLAB} må den skrives på rekursiv form \parencite[s.~53–58]{Dre2023Simulink}. Vi har i denne oppgaven valgt å benytte Euler's forovermetode som utrykkes som
\begin{equation}\label{eq:kap1_area_rec}
    y_k = y_{k-1} + \Delta{x_k} \cdot f(x_{k-1}) \quad \forall k=2,..., n
\end{equation}
hvor $k$ er indeksen til gjeldende element, $y_{k-1}$ er arealet under integranden ved forrige element, $\Delta{x_k}$ er lengden av intervallet ved gjeldende element og $f(x_{k-1})$ er verdien til integranden ved forrige element.

Kode~\ref{lst:kap1_euler} viser en generell implementering av ligning \eqref{eq:kap1_area_rec} i \textsc{MATLAB}. Legg merke til at en initialverdi for $y(k)$ må tilordnes for å kunne estimere arealet på det første intervallet i serien.

\subfile{./kode/kap1_euler_forward.tex}

Grunnen til at vi har valgt å benytte Euler's forovermetode at metoden gir et estimat på integralet ved gjeldende element $k$ gitt verdien av integralet $y(k-1)$ og integranden $f(k-1)$ ved forrige element. Dette er ønskelig fordi vi da kun må tilordne to verdier for å kunne estimere itegralet for hele serien.

\section{Implentering av numerisk integrasjon i \textsc{MATLAB}}

For å verifisere at Euler's forovermetode kan brukes til å integrere måledata i \textsc{MATLAB} har vi utført tre forsøk. Alle tre forsøkene tar utgangspunkt i å simulere fylling og tapping av væske i en beholder. En skisse av det konseptuelle oppsettet for forsøket kan sees i figur~\ref{fig:kap1_skisse_beholder}.

\subfile{./figurer/kap1_skisse_beholder.tex}

Raten $u$ det fylles og tappes væske i beholderen bestemmes av måledata fra lyssensoren til \textsc{EV3}-roboten. Volumet $y$ av væske i beholderen kalkulereres ved å numerisk integrere fylle- og tapperaten $u$ i \textsc{MATLAB}. Dette gjøres ved benytte implementeringen av Euler's fremovermetode vist i kode~\ref{lst:kap1_integrasjon_beholder}.

\subfile{./kode/kap1_integrasjon_beholder.tex}

For å verifisere at implementeringen vår i \textsc{MATLAB} av numerisk integrasjon er riktig har vi sammenlignet resultatene fra de tre forsøkene med analytisk kalkulerte verdier.

\subsection{Integrasjon av konstant}\label{sub:kap1_integrasjon_konstant}

Dette forsøket simulerer fylling og tapping med en konstant rate av væske i beholderen. Fylling og tapping av væske gjennom kranene blir simulert ved å lese av reflektert lys fra en gråskalastripe med lysmåleren til \textsc{EV3}-roboten. Gråskalastripen som ble brukt i forsøket kan sees i figur~\ref{fig:kap1_skala}.

\subfile{./figurer/kap1_skala.tex}

Lysmåleren til \textsc{EV3}-roboten ble plassert på gråskalafeltet i midten av stripen i figur~\ref{fig:kap1_skala}. Når filen \href{https://github.com/solbero/lego-project/blob/main/src/Prosjekt01_NumeriskIntegrasjon/Prosjekt01_NumeriskIntegrasjon.m}{\texttt{Prosjekt01\_NumeriskIntegrasjon.m}} ble kjørt ble den første lysmålingen fra lyssensoren gitt verdien $\SI{0}{cl/s}$. Ved å flytte lysmåleren til høyre for startfeltet vil strømningsraten $u(k)$ bli positiv, og ved å flytte lysmåleren til venstre for startfeltet vil strømningsraten $u(k)$ bli negativ.

I figur~\ref{fig:kap1_konstant} er fylle og tapperaten $u(t)$ plottet sammen med volumet $y(t)$ i beholderen.

\subfile{./grafer/kap1_konstant.tex}

For å kontrollere at implementeringen i \textsc{MATLAB} av Euler's fremovermetode er korrekt, kan vi sammenligne resultatene i figur~\ref{fig:kap1_konstant} med det analytisk integralet til $u(t)$. For at vår implementering av numerisk integrasjon skal være korrekt så skal integralet $I$ tilsvare arealet $A_1$ under $u(t)$ og forandringen i volum $\Delta y$ på intervallet $t$ = [\SI{1.8}{s}, \SI{4.4}{s}].
\begin{align}
    I & = \int_{1.8}^{4.4} \Delta u \, dt \nonumber       \\
      & = \Delta u \cdot t \, \Big|_{1.8}^{4.4} \nonumber \\
      & = \Delta u \cdot (4.4 - 1.8) \nonumber            \\
      & = \SI{52}{cl} \label{eq:kap1_konstant_integral}
\end{align}

Ved å sammeligne integralet $I$ i ligning \eqref{eq:kap1_konstant_integral} med figur~\ref{fig:kap1_konstant} kan vi verifisere at imlementeringen vår er korrekt for numerisk integrasjon av en konstant rate.

\subsection{Integrasjon av sinussignal}

Dette forsøket simulerer også fylling og tapping av væske i beholderen skissert i figur~\ref{fig:kap1_skisse_beholder}. Til forskjell fra forsøket beskrevet i seksjon~\ref{sub:kap1_integrasjon_konstant} simuleres nå fylling og tapping som et sinussignal.

Et sinussignal blir generert ved å bevege lysmåleren til \textsc{EV3}-roboten i en sirkel på et A4-gråskalaark som i figur~\ref{fig:kap1_ark}.

\subfile{./figurer/kap1_ark.tex}

To målinger ble gjennomført der lysmåleren ble beveget med konstant fart i en sirkel på gråskalaarket. Under første måling ble lysmåleren beveget med en lav fart, og under andre måling ble lysmåleren beveget med en høy fart.

Figur~\ref{fig:kap1_sinus1} viser fylle- og tapperaten $u_1(t)$ ved lav fart plottet sammen med volumet $y_1(t)$ i beholderen. Figur~\ref{fig:kap1_sinus2} viser fylle- og tapperaten $u_2(t)$ ved høy fart plottet sammen med volumet $y_2(t)$ i beholderen.

\subfile{./grafer/kap1_sinus1.tex}

\subfile{./grafer/kap1_sinus2.tex}

For å kontrollere at implementeringen i \textsc{MATLAB} av Euler's fremovermetode er korrekt for integrasjon av et sinussignal, kan vi sammenligne måledataene for lav og høy fart med det analytisk integralet av sinussignalet $u(t)$.

Det bestemte integralet av en sinus er
\begin{align}
    I & = \int_{0}^{t} a_u \cdot \sin{(\omega \cdot t)} \, dt \nonumber \\
      & = -\frac{a_u}{\omega} \cdot \cos{(\omega \cdot t)}
\end{align}
som gir oss at amplituden $a_y$ til volumet $y(t)$ tilsvarer amplituden $a_u$ til $u(t)$ delt på vinkelfrekvensen $\omega$.
\begin{equation}
    a_y = \frac{a_u}{\omega} \label{eq:kap1_amplitude}
\end{equation}

Tabell~\ref{tab:kap1_sinus} viser utregning av vinkelfreksvens $\omega$ og amplitudene $a_u$ og $a_y$ til $u(t)$ og $y(t)$ for høy og lav fart.

\subfile{./tabeller/kap1_sinus_tabell.tex}

Ved a sammenligne verdiene i tabell~\ref{tab:kap1_sinus} med verdiene lest av fra figur~\ref{fig:kap1_sinus1} og \ref{fig:kap1_sinus2} kan vi konkludere med at implementeringen av den numeriske integrasjonen er korrekt.

\subsection{Integrasjon av chirpsignal}

For å skjekke at amplituden $a_y$ til det integrerte signalet $y(t)$ er avhengig av vinkelfrekvensen $\omega$ til signalet $u(t)$ generete vi et chirpsignal med lyssensoren til \textsc{EV3}-roboten. Chirpsignalet ble generert ved å jevnt øke farten som lysmåleren ble beveget i sirkel som sett i figur~\ref{fig:kap1_ark}.

Hvis amplituden til $y(t)$ er avhengig av vinkelfrekvensen $\omega$ til $u(t)$ som vist i ligning \eqref{eq:kap1_amplitude}, så skal amplituden til $y(t)$ synke med økende vinkelfrekvens hvis amplituden $a_u$ til $u(t)$ er konstant.

Figur~\ref{fig:kap1_chirp} viser fylle- og tapperaten $u(t)$ plottet sammen med volumet $y(t)$ i beholderen. Figuren viser at amplituden $a_y$ til $y(t)$ synker med økende vinkelfrekvens i signalet $u(t)$, noe som betyr at implementeringen er korrekt.

\subfile{./grafer/kap1_chirp.tex}

\end{document}