%%%%%%%%%%%%%%%%%%%%%%%%%%%%%%%%%%%%%%%%%%%%%%%%%%%%%%%%%%%%%%%
% KAPPITTEL 1: NUMERISK INTEGRASJON
%%%%%%%%%%%%%%%%%%%%%%%%%%%%%%%%%%%%%%%%%%%%%%%%%%%%%%%%%%%%%%%

\documentclass[main.tex]{subfiles}

%%%%%%%%%%%%%%%%%%%%%%%%%%%%%%%%%%%%%%%%%%%%%%%%%%%%%%%%%%%%%%%
% Start av dokumentet
%%%%%%%%%%%%%%%%%%%%%%%%%%%%%%%%%%%%%%%%%%%%%%%%%%%%%%%%%%%%%%%

\begin{document}

\chapter{Numerisk integrasjon (utført av hele gruppen) }\label{kap:integrasjon}

\section{Problemstilling}
Gruppen ønsker å vise at måledata fra \textsc{EV3}-roboten kan numerisk integreres ved bruk av \textsc{MATLAB}.

\section{Numerisk integrasjon}
Numerisk integrasjon er en metode for å estimere et bestemt integral. Metoden bygger på at integranden $f(x)$ er, eller kan deles opp i, en serie med diskrete intervaller. Arealet under $f(x)$ for serien estimeres som en sum av arealet for hvert intervall i serien \parencite[s. 297–301]{AdaEss2017}.

Utrykket for å estimere arealet under $f(x)$ for en serie diskrete verdier $n$ kan derfor utrykkes som
\begin{equation}\label{eq:kap1_sum_area}
    \hat{A} =\sum_{k=1}^{n-1} \, f(x_k) \cdot \Delta x_k
\end{equation}
hvor $\Delta x_k$ er lengden på intervallet mellom to elementer i serien og $f(x_k)$ er verdien til integranden på det respektive intervallet. Figur~\ref{fig:kap1_partitioning} viser et grafisk eksempel på hvordan ligning \eqref{eq:kap1_sum_area} estimerer integralet til $f(x)$.

\subfile{./figurer/kap1_partitioning.tex}

Det bør bemerkses at etterhvert som intervallene blir uendelig små, altså at $\lim_{n \to \infty}$, så vil ligning \eqref{eq:kap1_sum_area} konvergere mot det bestemte integralet for $f(x)$.

For å kunne benytte ligning \eqref{eq:kap1_sum_area} i \textsc{MATLAB} må den skrives på rekursiv form \parencite[s.~53–58]{Dre2023Simulink}. Vi har i denne oppgaven valgt å benytte Euler's forovermetode
\begin{equation}\label{eq:kap1_area_rec}
    y_k = y_{k-1} + \Delta{x_k} \cdot f(x_{k-1}) \quad \forall k=2,..., n
\end{equation}
hvor $k$ er indeksen til gjeldende element, $y_{k-1}$ er arealet under integranden ved forrige måling, $\Delta{x_k}$ er lengden av intervallet ved gjeldende element og $f(x_{k-1})$ er verdien til integranden ved forrige element.

En generell implementering av den rekursive ligningen \eqref{eq:kap1_area_rec} i \textsc{MATLAB} kan sees i kode~\ref{lst:kap1_euler}. Legg merke til at en initialverdi for $y(k)$ må tilordnes for å kunne estimere arealet på det første intervallet.

\subfile{./kode/kap1_euler_forward.tex}

Grunnen til at vi har valgt å benytte Euler's forovermetode at metoden gir et estimat på integralet ved gjeldende element gitt verdien av integralet og integranden ved forrige element. Dette er ønskelig fordi vi da kun må lagre to verdier for å kunne estimere itegralet for hele serien.

\section{Implentering av numerisk integrasjon i \textsc{MATLAB}}

For å verifisere at Euler's forovermetode kan brukes til å integrere måledata i \textsc{MATLAB} har vi utført to forsøk. Begge forsøkene tar utgangspunkt i å simulere fylling og tapping av væske i en beholder. En skisse av simuleringen for forsøket kan sees i figur~\ref{fig:kap1_skisse_beholder}.

\subfile{./figurer/kap1_skisse_beholder.tex}

Raten $u(k)$ det fylles og tappes væske i beholderen styres av måledata fra lyssensoren til \textsc{EV3}-roboten. Volumet av væske i beholderen $y(k)$ kalkulereres ved å numerisk integrere fylle- og tapperaten i \textsc{MATLAB}. Dette gjøres ved benytte implementering vist i kode~\ref{lst:kap1_integrasjon_beholder} av den generelle koden vist i kode~\ref{lst:kap1_euler}.

\subfile{./kode/kap1_integrasjon_beholder.tex}

For å verifisere at implementeringen vår i \textsc{MATLAB} er riktig har vi sammenlignet resultatene fra numerisk integrasjonen i de to forsøkene med analytiske verdier.

\subsection{Integrasjon av en konstant}\label{sub:kap1_integrasjon_konstant}

Dette forsøket simulerer fylling og tapping av væske i beholderen i figur~\ref{fig:kap1_skisse_beholder} med en konstant rate. Fylling og tapping av væske gjennom kranene blir simulert ved å lese av reflektert lys fra en gråskalastripe. Gråskalastripen kan sees i figur~\ref{fig:kap1_skala}.

\subfile{./figurer/kap1_skala.tex}

Lysmåleren til \textsc{EV3}-roboten blir plasert på gråskalafeltet i midten av stripen i figur~\ref{fig:kap1_skala}. Når filen \href{https://github.com/solbero/lego-project/blob/main/src/Prosjekt01_NumeriskIntegrasjon/Prosjekt01_NumeriskIntegrasjon.m}{\texttt{Prosjekt01\_NumeriskIntegrasjon.m}} blir kjørt blir den første lysmålingen fra lyssensoren gitt verdien $\SI{0}{cl/s}$. Ved å flytte lysmåleren til høyre for startfeltet vil strømningsraten bli positiv, og ved å flytte lysmåleren til venstre for startfeltet vil strømningsraten bli negativ.

I figur~\ref{fig:kap1_konstant} er fylle og tapperaten $u(k)$ plottet sammen med volumet $y(k)$ i beholderen.

\subfile{./grafer/kap1_konstant.tex}

For å kontrollere at implementeringen i \textsc{MATLAB} av Euler's fremovermetode er korrekt, kan vi sammenligne resultatene i figur~\ref{fig:kap1_konstant} med det analytisk integralet til $u(t)$. Ved å integrere $u(t)$ fra $t = \SI{1.8}{s}$ til $t = \SI{4.4}{s}$
\begin{align}
    I & = \int_{1.8}^{4.4} \Delta u \, dt \nonumber       \\
      & = \Delta u \cdot t \, \Big|_{1.8}^{4.4} \nonumber \\
      & = \Delta u \cdot (4.4 - 1.8) \nonumber            \\
      & = \SI{52}{cl} \label{eq:kap1_konstant_integral}
\end{align}
For at vår implementering av numerisk integrasjon skal være korrekt så skal integralet $I$ tilsvare arealet under $u(t)$ og forandringen i volum $y(t)$ på intervallet $t = \SI{1.8}{s}$ til $t = \SI{4.4}{s}$. Ved å sammeligne integralet i ligning \eqref{eq:kap1_konstant_integral} med figur~\ref{fig:kap1_konstant} kan vi verifisere at imlementeringen vår er korrekt for numerisk integrasjon av en konstant rate.

\subsection{Integrasjon av et sinussignal}

Dette forsøket simulerer også fylling og tapping av væske i beholderen skissert i figur~\ref{fig:kap1_skisse_beholder}. Til forskjell fra forsøket beskrevet i seksjon~\ref{sub:kap1_integrasjon_konstant} simuleres nå fylling og tapping som et sinus.

En sinus blir generert ved å bevege lysmåleren til \textsc{EV3}-roboten i en sirkel på et A4-gråskalaark som kas sees i figur~\ref{fig:kap1_ark}.

\subfile{./figurer/kap1_ark.tex}

To målinger ble gjennomført der lysmåleren ble beveget med konstant fart en sirkel på gråskalaarket. Under første måling ble lysmåleren beveget med en lav fart, og under andre måling ble lysmåleren beveget med en høy hastighet.

I figur~\ref{fig:kap1_sinus1} er fylle- og tapperaten $u_1(t)$ plottet sammen med volumet $y_1(t)$ i beholderen for den første hastigheten. Figur~\ref{fig:kap1_sinus2} viser fylle- og tapperaten $u_2(t)$ plottet sammen med volumet $y_2(t)$ i beholderen for den andre hastigheten.

\subfile{./grafer/kap1_sinus1.tex}

\subfile{./grafer/kap1_sinus2.tex}

For å kontrollere implementeringen av Euler's fremovermetode korrekt, kan vi sammenligne måledataene med et analytisk integral av et sinussignal.

Integralet av et sinussignal kan skrives som
\begin{align}
    I & = \int_{0}^{t} a_u \cdot \sin{(\omega \cdot t)} \, dt \nonumber \\
      & = -\frac{a}{\omega} \cdot \cos{(\omega \cdot t)}
\end{align}

hvor vinkelfrekvensen $\omega$ regnes ut fra perioden $T_p$ lest av fra figur~\ref{fig:kap1_sinus1} og \ref{fig:kap1_sinus2}
\begin{equation}
    \omega = \frac{2\pi}{T_p}
\end{equation}

som betyr at amplituden $a_y$ til volumet $y(t)$ i beholderen kan regnes ut fra
\begin{equation}
    a_y = \frac{a_u}{\omega}
\end{equation}

Tabell~\ref{tab:kap1_sinus} viser disse utregningene for lav og høy hastighet på sirkelbevegelsen.

\subfile{./tabeller/kap1_sinus.tex}

Ved a sammenligne verdiene for $a_y$ i tabellen med verdiene lest av fra figur~\ref{fig:kap1_sinus1} og \ref{fig:kap1_sinus2} kan vi konkludere med at implementeringen av den numeriske integrasjonen er korrekt.

\subsection{Integrasjon som funksjon}

\end{document}
