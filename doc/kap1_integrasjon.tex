%%%%%%%%%%%%%%%%%%%%%%%%%%%%%%%%%%%%%%%%%%%%%%%%%%%%%%%%%%%%%%%
% KAPPITTEL 1: NUMERISK INTEGRASJON
%%%%%%%%%%%%%%%%%%%%%%%%%%%%%%%%%%%%%%%%%%%%%%%%%%%%%%%%%%%%%%%

\documentclass[./main.tex]{subfiles}

%%%%%%%%%%%%%%%%%%%%%%%%%%%%%%%%%%%%%%%%%%%%%%%%%%%%%%%%%%%%%%%
% Start av dokumentet
%%%%%%%%%%%%%%%%%%%%%%%%%%%%%%%%%%%%%%%%%%%%%%%%%%%%%%%%%%%%%%%

\begin{document}

\chapter{Numerisk integrasjon (utført av hele gruppen) }\label{kap:integrasjon}

Overskriftene som er brukt i denne malen er kun forslag. Du må gjerne
endre de til det du mener passer bedre.

\section{Problemstilling}

Husk at det er smart å bruke oversiktsfigurer/bilder som illustrerer
problemstillingen på en slik måte at leseren enkelt forstår hva det handler
om. Et bilde sier mer enn 1000 ord.

Bruk gjerne bilder fra selve prosjektbeskrivelsen, men husk på å sitere/referere
til denne. Et eksempel på en slik sitering er vist under.

For mer tips om referering, se dokumentet \fbox{Notat 4, Introduksjon
  til Overleaf og LaTeX.pdf}.

\section{Forslag til løsning}

Husk at det er smart å beskrive løsningen ved å bruke matematiske
formler. Ikke skriv formlene som endel av teksten, men bruk
\fbox{\tt equation}-funksjonen for dette.

Ligninger er kompakte informasjonsbærere, så istendenfor å
beskrive en sammenheng med ord bør du heller bruke en ligning.
Husk, en ligning sier mer enn 100 ord.

Inkluder gjerne en figur som viser detaljert informasjon om løsningen
og hvor du har inkludert  variabelnavn fra ligningen(e).

\section{Matlab- og Pythonkode}
Ta med de viktigste kodesnuttene. De som viser hvordan matematikken er
implementert.

Det kan være at det for noen prosjekt er mer naturlig å inkludere kode
sammen med matematikken i ``Forslag til løsning''. Det er ingen fasit
på dette.

\section{Resultat}

\subsection{Integrasjon av en konstant}

\subsection{Integrasjon av et sinussignal}

\end{document}
