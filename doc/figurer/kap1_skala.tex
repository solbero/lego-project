%%%%%%%%%%%%%%%%%%%%%%%%%%%%%%%%%%%%%%%%%%%%%%%%%%%%%%%%%%%%%%%
% KAPPITTEL 1: FIGUR SKISSE GLASS
%%%%%%%%%%%%%%%%%%%%%%%%%%%%%%%%%%%%%%%%%%%%%%%%%%%%%%%%%%%%%%%

\documentclass[../main.tex]{subfiles}

%%%%%%%%%%%%%%%%%%%%%%%%%%%%%%%%%%%%%%%%%%%%%%%%%%%%%%%%%%%%%%%
% Start av dokumentet
%%%%%%%%%%%%%%%%%%%%%%%%%%%%%%%%%%%%%%%%%%%%%%%%%%%%%%%%%%%%%%%

\begin{document}

\begin{figure}[H]
    \centering
    \includegraphics[width=0.8\textwidth]{\subfix{kap1_skala.png}}
    \caption{Glass som fylles med vann innebærer positive verdier på $u(k)$, mens drikking via sugerør innebærer negative verdier. Hentet fra \citetitle{Dre2023Lego} av \Citeauthor{Dre2023Lego}, \citeyear*{Dre2023Lego}, s. 75.}
    \label{fig:kap1_skala}
\end{figure}

\end{document}