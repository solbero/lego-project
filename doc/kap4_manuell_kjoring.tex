%%%%%%%%%%%%%%%%%%%%%%%%%%%%%%%%%%%%%%%%%%%%%%%%%%%%%%%%%%%%%%%
%
%%%%%%%%%%%%%%%%%%%%%%%%%%%%%%%%%%%%%%%%%%%%%%%%%%%%%%%%%%%%%%%

\documentclass{main.tex}[subfiles]

\begin{document}

\chapter{Manuell kjøring av Lego-robot (utført av hele gruppen)}\label{kap:manuell_kjoring}

\section{Problemstilling}

\section{Forslag til løsning}

\section{Resultat}

For at du skal slippe å lage tabellen selv, ligger den klar i malen
med ett eksempel på hvordan du kan utheve beste resultat.
\begin{table}[H]
  \centering
  \renewcommand{\arraystretch}{1.2}
  \caption{Sluttresultater fra manuell kjøring til deltagere i gruppe 20XX.}
  \hspace*{0mm}  \begin{tabular}{|c|c|c|c|c|}   \hline
                                 & Per              & Pål & Anne & Eva \\\hline\hline
    Plattform                    &                  &     &      &     \\\hline
    Strømkilde                   &                  &     &      &     \\\hline
    Samtidig plotting?           &                  &     &      &     \\\hline
    {\tt Referanse}              &                  &     &      &     \\\hline
    middelverdi $\mu$            &                  &     &      &     \\\hline
    |{\tt Referanse}-$\mu$|      &                  &     &      &     \\\hline
    standardavvik $\sigma$       &                  &     &      &     \\\hline
    kjøretid [sek]               &                  &     &      &     \\\hline
    IAE                          & {\bf 49.8}$^{*}$ &     &      &     \\\hline
    MAE                          &                  &     &      &     \\\hline
    TV\_A                        &                  &     &      &     \\\hline
    TV\_B                        &                  &     &      &     \\\hline
    middelverdi av $T_{s}$ [sek] &                  &     &      &     \\\hline
    antall målinger ($k$)        &                  &     &      &     \\\hline
  \end{tabular}    \label{tab:tall}
\end{table}

\end{document}